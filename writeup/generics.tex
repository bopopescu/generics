\documentclass[a4paper,10pt]{article}

\setcounter{secnumdepth}{5} % seting level of numbering (default for "report" is 3). With ''-1'' you have non number also for chapters
\setcounter{tocdepth}{5} % if you want all the levels in your table of contents

\usepackage{ucs}
\usepackage{amsfonts}
\usepackage{amssymb}
\usepackage{fontenc}
\usepackage{graphicx}

\usepackage[dvips]{hyperref}

\author{Erin Bennett}
\title{Notes on scalar adjectives}

\begin{document}
 
\tableofcontents

\section{Experiment 3A}
  \subsection{Motivation}
    Confirm that the examples of generic sentences that we think are weird are actually weird.
  \subsection{Method}
    \subsubsection{Participants} 20 participants were recruited over Amazon’s Mechanical Turk.
    \subsubsection{Procedure and materials} We took some generic sentences (e.g. As are B) from Cohen (1999) and my own brainstorming that seem to be false even though the majority of As are B. We asked participants ``how true'' they thought these sentences were on a scale from ``complete untrue'' to ``completely true''.

    \begin{center}
      \includegraphics[width=0.8\textwidth]{3a.png}
    \end{center}

  \subsection{Results}
    Generic utterances ``Birds fly'' and ``Mosquitos carry West Nile Virus'' and ``Lions have manes'' do indeed appear to be more true than the examples of bad generics ``Animals weigh less than 2 tons'', ``People have black hair'', ``People are over three years old'', ``People care about the environment'', ``Books are paperbacks'', and ``Shoes are size 7 and above''. However, I'm seeing a very gradient response pattern and there's a lot of variance in how true people think these sentences are.

    \begin{center}
      \includegraphics[width=\textwidth]{../why_style/examples3A/mturk/3a.png}
    \end{center}
    
    The pattern of responses is the same whether z-scored or not.

    \begin{center}
      \includegraphics[width=\textwidth]{../why_style/examples3A/mturk/3a-z.png}
    \end{center}
    
    \paragraph{Open questions}

\end{document}
